%%%%%%%%%%%%%%%%%%%%%%%%%%%%%%%%%%%%%%%%%%%%%%%%%%%%%%%
% A template for Wiley article submissions.
% Developed by Overleaf. 
%
% Please note that whilst this template provides a 
% preview of the typeset manuscript for submission, it 
% will not necessarily be the final publication layout.
%
% Usage notes:
% The "blind" option will make anonymous all author, affiliation, correspondence and funding information.
% Use "num-refs" option for numerical citation and references style.
% Use "alpha-refs" option for author-year citation and references style.

\documentclass[num-refs,twocolumn]{layout}
% \documentclass[blind,alpha-refs]{wiley-article}

% Add additional packages here if required
\usepackage{siunitx}
\usepackage{acronym}
\usepackage{epsfig}
\usepackage{caption}
\usepackage{float}
\usepackage{subcaption}
\usepackage{booktabs}
\usepackage{multirow}

\sisetup{
  round-mode          = places, % Rounds numbers
  round-precision     = 1, % to 2 places
}

% Update article type if known
\papertype{Original Article}
% Include section in journal if known, otherwise delete
\paperfield{Journal Section}

\title{A digital-based design methodology for the optimization of high-performance multi-material structures}

% List abbreviations here, if any. Please note that it is preferred that abbreviations be defined at the first instance they appear in the text, rather than creating an abbreviations list.
\abbrevs{}

% Include full author names and degrees, when required by the journal.
% Use the \authfn to add symbols for additional footnotes and present addresses, if any. Usually start with 1 for notes about author contributions; then continuing with 2 etc if any author has a different present address.
\author[1]{W. Zschiebsch}%\authfn{1}
\author[2]{A. Filippatos}
\author[1]{R. Böhm}

%\contrib[\authfn{1}]{Equally contributing authors.}
% Include full affiliation details for all authors
\affil[1]{Leipzig University of Applied Sciences, Faculty of Engineering, Germany}
\affil[2]{Dresden Center for Intelligent Materials, Germany}
\corraddress{Willi Zschiebsch, Leipzig University of Applied Sciences, Faculty of Engineering, Germany}
\corremail{willi.zschiebsch@gmail.com}%????????????????
%\fundinginfo{Funder One, Funder One Department, Grant/Award Number: 123456, 123457 and 123458; Funder Two, Funder Two Department, Grant/Award Number: 123459}%DELETE ????????????????
% Include the name of the author that should appear in the running header
\runningauthor{Zschiebsch et al.}%????????????????

\begin{document}

\acrodef{fdm}[FDM]{finite difference method}
\acrodef{fe}[FE]{finite element}
\acrodef{fea}[FEA]{finite element analysis}
\acrodef{fem}[FEM]{finite element method}
\acrodef{nsa}[NSA]{normal stress averaging}
\acrodef{pia}[PIA]{principle of independent action}

\begin{frontmatter}
\maketitle

\begin{abstract}
% Vorschlag Felix:
To determine the probability of failure for the brittle materials, the 'principle of independent action' (pia) method can be used with stresses calculated by Finite Element (FE) simulations. In this work different methods for implementing the pia method into existing fem programs are evaluated in terms of computational error and effort. It could be shown, that the commonly used normal stress averaging (NSA) should not be used and should replaced with high order integration techniques. Further, the effect of different Weibull parameters on the performance as well as a method to artificially increase the precision are shown in this paper. The results can be used to determine the best method for a particular use case scenario.

% "alte Version"
Evaluating components using \ac{fem} based algorithms is a standard practise and is
therefore also used for the evaluation of brittle materials. In this work, different 
methods for implementing the \ac*{pia} method into existing \ac{fem} programs are evaluated. Their advantages and disadvantages, mainly the computational error and effort, are analyzed and compared by numerical simulations. 
In particular, it shows that the commonly used \ac*{nsa} approach should not be used and should replaced with high order integration techniques.
Further, the effect of different Weibull parameters on the performance as well as a method to artificially increase the precision are shown in this paper. The results can be used to determine the best method for a particular use case scenario.
% Please include a maximum of seven keywords
\keywords{brittle, failure, FEM, isoparametric, PIA, strength, Weibull theory}
\end{abstract}
\end{frontmatter}

%%%%%%%%%%%%%%%%%%%%%%%%%%%%%%%%%%%%%%%%%%%%%%%%%%%%%%%%%%%%%%%%%%%%%%%%%%%%%%%%%%%%%%%%%%%%%%%%%%%%%%%%%%

\section{State of the Art}

Later Barnett \cite{barnett1978} and Freudenthal \cite{freudenthal1968} extended 
 Weibull's formula to also treat uniform multi-axial stresses \cite{lamon2016}. 
\section{Method}
From the beginning of the design process when each component, 
like part specific functions are briefly defined, the system interactions 
are described as a directed graph.
During further refinements in the design process, 
these components are changed by using methods like be 
subdivision, summariz or outsources.
The individual components therefore have to implement a certain interface, 
which allows to set the inputs and updates the outputs accordingly.

Using a directed graph method to represent each process Each step in a workflow can be represented as blackbox with a certain input and output (fig. \ref{pic:process-single}).
Due to simplicity the inputs and outputs have values that consists of a list of numbers. 
This values may be connected to other steps in the workflow resulting in different dependencies.
\begin{figure}[h]
    \centering
    \includegraphics[scale=0.5]{pics/VDI_2206.PNG}
    \caption{\label{pic:VDI2206} "General approach for development and construction (VDI 2221)." \cite{Jansch2006THEDO}}
\end{figure}\\
test
\begin{algorithm}[ht]
    \SetAlgoLined
    \KwIn{list of porcesses}
    \KwOut{sortet list of processes}
    use bubble sort to sort the list\; 
    \Fn{\FBSort{Process A, Process B}}{
        \eIf{A.InputConnections == B.InputConnections}{
            \eIf{A.Time == B.Time}{
                \eIf{A.InputConnections-A.InputChanges == B.InputConnections-B.InputChanges}{
                    \eIf{A.InputChanges == B.InputChanges}{
                        \Return{A.OutputConnections < B.OutputConnections}
                    }{
                        \Return{A.InputChanges > B.InputChanges}
                    }
                }{
                    \Return{A.InputConnections-A.InputChanges < B.InputConnections-B.InputChanges}
                }
            }{
                \Return{A.Time < B.Time}
            }
        }{
            \Return{A.InputConnections < B.InputConnections}
        }
    }
    \textbf{end}
\caption{\label{code:sort-process}Sort-Algorithm}
\end{algorithm}
text
\begin{algorithm}[bt]
    \KwIn{List of Processes\\
    Priority $\in$ ["manual", "dynamic", "time"]}
    \KwOut{Updated and converged processes}
    sort(List of Processes)\;
    \textbf{set} GState to "changed"\;
    \textbf{set} StartPos to 0\;
    \While{GState = "changed"}{
        \textbf{set} GState to "unchanged"\;
        \For{i = StartPos to sizeof(List of Processes)}{
            \textbf{set} CurrProcess as ith Element of List of Processes
            \textbf{set} GState to "unchanged"\;
            \textbf{set} STime to currentTime()\;
            \textbf{set} PState to CurrProcess.update()\;
            \textbf{set} ETime to currentTime()\;
            \textbf{set} $\Delta$Time to ETime - STime\;
            \If{Priority equals "time"}{
                CurrProcess.Time = round($\log_{10}${$\Delta$Time})\;
            }
            \If{PState equals "changed"}{
                \textbf{set} GState to "changed"\;
            }
            \If{CurrProcess.Inputs equals 0 \emph{\textbf{and}} Priority not "manuel"}{
                \textbf{set} StartPos to i\;
            }
            \If{Priority equals "dynamic" \emph{\textbf{or}} Priority equals "time"}{
                \textbf{set} List of FProcesses as List of Processes starting from CurrProcess\;
                \ForEach{FurtherProcess in  List of FProcesses}{
                    FurtherProcess.InputChanges.update()\;
                }
                sort(List of FProcesses)\;
            }
            increase i by one\;
       }
    }
\caption{\label{code:update-standard-processor}Update-Algorithm of the standard processor}
\end{algorithm}\\
Test
Test
test
\begin{figure}[h]
    \centering
    \includegraphics[scale=0.5]{pics/evol_alg.PNG}
    \caption{\label{pic:evol_alg} "General approach for development and construction (VDI 2221)." \cite{Jansch2006THEDO}}
\end{figure}\\
test

\section{Result}

\begin{figure}[h]
    \centering
    \includegraphics[scale=0.5]{pics/VDI_2206.PNG}
    \caption{\label{pic:VDI2206} "General approach for development and construction (VDI 2221)." \cite{Jansch2006THEDO}}
\end{figure}

ahnsdkjhauish
\input{src/discussion.tex}
\section{Conclusion}
Designing multifunctional and lightweight products is time consuming work, due to a high number of functional dependencies that have to be taken into account.
In this article a digital methodology is introduced, 
which aims to reduce the development time and 
increases the overall product quality by enabling multi domain optimizations.
The introduced system used the advantages of a complete digitizes, graph-based process representation.
It enables specialists to collaborate on problems with multiple domains and allows to outsource time-consuming operations to specialized servers.\\
With an heuristic approach the software tries to compute a stable configuration of the modeled interactions.
In the first example, this capability was used to solve the complex interactions of design problem which has to take the eigenweight into account.
It showed how different design process could be linked together, without introducing a new file format.\\
Additionally, given a ranking of multiple solution is described by a fitness function, the developed software can find the global optimum.
This was proven with the second example, where the system found the optimal layer constellation of an high-speed glass fiber reinforced plastics rotor.
It further demonstrated how multiple optimization goals can be taken into account.\\
While the test showed room for improvements in terms of performance, stability and user friendliness,
they also shows a major potential for great work relief and time savings.

\section*{List of Symbols}
\begin{table}[H]
    %\caption{\label{tb:los} A list of coordinates, which was used to generate the different degrees of deformation.}
    \centering
\begin{tabular}{c | c } 
    \toprule
    \textbf{symbol} & \textbf{meaning} \\ [0.5ex] 
    \midrule
    h & height\\
    J & Jacobian matrix\\
    m & Weibull module\\
    N & interpolation matrix\\
    NoE & number of elements\\
    NoN & number of nodes\\
    n & shape function\\
    $\sigma$ & stress\\
    V & Volume\\ 
    x, y & global coordinates\\
    $\xi$, $\eta$ & natural/local coordinates\\[1ex]
    \bottomrule
\end{tabular}
\end{table}

\section*{List of Indices}
\begin{table}[H]
    \centering
\begin{tabular}{c | c} 
    \toprule
\textbf{index} & \textbf{meaning} \\ [0.5ex] 
\midrule
$eff$ & effective\\
$f$ & rapture or failure\\
i & element number\\
j & node index in element\\
k & index of principle stress\\
$max$ & maximum\\ [1ex] 
\bottomrule
\end{tabular}
\end{table}

%\section*{acknowledgements}
%Acknowledgements should include contributions from anyone who does not meet the criteria for authorship (for example, to recognize contributions from people who provided technical help, collation of data, writing assistance, acquisition of funding, or a department chairperson who provided general support), as well as any funding or other support information.
%\printendnotes%???????????????????????????
% Submissions are not required to reflect the precise reference formatting of the journal (use of italics, bold etc.), however it is important that all key elements of each reference are included.
\bibliography{sources}
%\include{appendix}
\end{document}
