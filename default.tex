%%%%%%%%%%%%%%%%%%%%%%%%%%%%%%%%%%%%%%%%%%%%%%%%%%%%%%%
% A template for Wiley article submissions.
% Developed by Overleaf. 
%
% Please note that whilst this template provides a 
% preview of the typeset manuscript for submission, it 
% will not necessarily be the final publication layout.
%
% Usage notes:
% The "blind" option will make anonymous all author, affiliation, correspondence and funding information.
% Use "num-refs" option for numerical citation and references style.
% Use "alpha-refs" option for author-year citation and references style.

\documentclass[num-refs,twocolumn]{wiley-article}

% \documentclass[blind,alpha-refs]{wiley-article}

% Add additional packages here if required
\usepackage{siunitx}
\usepackage{acronym}
\usepackage{epsfig}
\usepackage{caption}
\usepackage{float}
\usepackage{subcaption}
\usepackage{booktabs}
\usepackage{multirow}

\sisetup{
  round-mode          = places, % Rounds numbers
  round-precision     = 1, % to 2 places
}

% Update article type if known
\papertype{Original Article}
% Include section in journal if known, otherwise delete
\paperfield{Journal Section}

\title{A digital-based1 design methodology for the optimization of high-performance multi-material structures}

% List abbreviations here, if any. Please note that it is preferred that abbreviations be defined at the first instance they appear in the text, rather than creating an abbreviations list.
\abbrevs{}

% Include full author names and degrees, when required by the journal.
% Use the \authfn to add symbols for additional footnotes and present addresses, if any. Usually start with 1 for notes about author contributions; then continuing with 2 etc if any author has a different present address.
\author[1]{W. Zschiebsch}%\authfn{1}
\author[2]{A. Filippatos}
\author[1]{R. Böhm}

%\contrib[\authfn{1}]{Equally contributing authors.}
% Include full affiliation details for all authors
\affil[1]{Leipzig University of Applied Sciences, Faculty of Engineering, Germany}
\affil[2]{Dresden Center for Intelligent Materials, Germany}
\corraddress{Willi Zschiebsch, Leipzig University of Applied Sciences, Faculty of Engineering, Germany}
\corremail{willi.zschiebsch@gmail.com}%????????????????
%\fundinginfo{Funder One, Funder One Department, Grant/Award Number: 123456, 123457 and 123458; Funder Two, Funder Two Department, Grant/Award Number: 123459}%DELETE ????????????????
% Include the name of the author that should appear in the running header
\runningauthor{Zschiebsch et al.}%????????????????

\begin{document}

\acrodef{fdm}[FDM]{finite difference method}
\acrodef{fe}[FE]{finite element}
\acrodef{fea}[FEA]{finite element analysis}
\acrodef{fem}[FEM]{finite element method}
\acrodef{nsa}[NSA]{normal stress averaging}
\acrodef{pia}[PIA]{principle of independent action}

\begin{frontmatter}
\maketitle

\begin{abstract}
A central process in composite lightweight engineering is the design of different fibre-reinforced parts. Some sub-processes such as the construction and numerical failure analysis as well as certain parts of the design process itself can be significantly improved using modern software tools. This often means that a compromise between optimization and increasing development costs must be found in order to balance structural complexity, number of design iteration loops and subsequent changes in the requirements.
We introduce a computer-driven automation process for a multi-domain, parameter-driven design optimization. The proposed concept was built around the idea that the methodology can be used with different software tools that are already in operation in the design process of lightweight structures and therefore allows an easy implementation in already existing development chains.
The developed process was successfully applied to different design scenarios, for example for designing GFRP rotors with respect to their structural dynamics performance. The results show promising time savings during the design phase and allow to quickly adapt the design to subsequent changes in the optimization goal.
\keywords{Composites, Optimization, Design and Modelling, Smart by Design, Design Automation}
\end{abstract}
\end{frontmatter}

%%%%%%%%%%%%%%%%%%%%%%%%%%%%%%%%%%%%%%%%%%%%%%%%%%%%%%%%%%%%%%%%%%%%%%%%%%%%%%%%%%%%%%%%%%%%%%%%%%%%%%%%%%

\section{\label{sec:SoA}Introduction}
Developing products, especially lightweight structures, 
requires variety of steps to come from the original idea to the final product.
One approach to follow these steps to result in a successful production 
are described in the guidelines of VDI 2206 \cite{gausmeier2002} and VDI 2221 \cite{Jansch2006THEDO}.
These guidelines give a general methodology for designing technical systems and products, 
by introducing a methodical and systematic designing procedure, to work with maximum efficiency.
Since the first introduction in 1993 these guidelines have been applied within mechanical engineering, precision mechanics, 
switches and software development and the planning of process engineering \cite{pahl_beitz_2013}. 
Depending on the individual outcome steps might be repeated in as part of cyclic quality enforcing technique.
Especially in lightweight design the analysis, design and manufacturing processes are so strong connected, 
that they all have to be considered in every phase of the development process (fig. \ref{pic:interactive-design}).
The amount of cyclic repeats do increase with higher demands on the products efficiency and error margin.
These causes it to be one of the main factors for increased development time and 
also reduces the amount of reusable components with slightly changes in the product demand.\\
\begin{figure}[h]
    \centering
    \includegraphics[scale=0.4]{pics/interactive-design.PNG}
    \caption{\label{pic:interactive-design} "A spiral development approach for complex function‐integrative systems." \cite{Modler2020}}
\end{figure}\\
Further, with increasing complexity the number of required experts increases.
This often can lead to miss-communication between the experts themselves or in translation problems 
to transfer information form one system to another (CAD-chains).\\
Using the advantages of digital technologies is since the dawn of computer aided design (CAD) in the mid-1960s
a lot of research has been done to solve these issues. 
Resulting in row of CAD-systems, Simulation programs and product data management (PDM) systems.
But especially notable was E. Allen, who showed in 1970, that these kind of systems, 
can be described as a directed graph, consisting of a set of nodes and directed edges, 
which are connecting the nodes with each other \cite{allen_control_1970}.
Each node hereby describes a linear sequence of program instructions, 
what means that nodes have one entry point (the first instruction executed) and 
one exit point (the last instruction executed).
Following along the edges, in a directed graph, results in a series of blocks what is called a path.
This description allows to also describe closed path or circuit is a paths, 
where the path ends at the same start block. 
\cite{allen_control_1970}\\
Problems that can be described with these models are summarized 
under the term "directed graph theory" \cite{bang-jensen_digraphs_2009, lehman_directed_2010}
and have been studied for a wide variety of different applications in 
\cite{lehman_directed_2010, aho_theory_1972, kam_global_1976}.\\
Graph based solutions are often characterized as easy to understand and to compute, 
but also have difficulties if the causality can change dynamically \cite{sinha_modeling_2001}. 
Further, with systems like CAMP-G and SIDOPS+ \cite{breunese_modeling_1996} mechanism to solve 
nonlinear high-dimensional graph models that can contain both continuous-time and 
discrete-time parts have been developed.\\
Today, Graph-Based Approaches has been implemented in a handful of design programs and 
simulations tools \cite{noauthor_dynamo_2020, noauthor_function_2020, noauthor_systems_2020},
but none of these implementations allows the use of processes of other design or simulation tools.\\
Complex multi-disciplinary systems requires the expertise of a group of collaborating specialists:
Designers with backgrounds in different disciplines collaborate with analysts, manufacturing engineers, marketing
specialists, and business managers.\cite{sinha_modeling_2001}\\
Therefor computer aided engineering (CAE) technologies, that provide sharing, visualization, 
documentation, and management of product models,
are used to coordinate design processes among geographically dispersed and multidisciplinary
teams \cite{finger_creating_1994, bajaj_web_1999, iwasaki_web-based_2002}.
However, the aspect of collaborative simulation modeling is still in its infancy. 
The current approach to support collaborative modeling, is done by shared model representations, 
repositories to manage model components, and model abstraction capabilities to provide different views of models to designers.
Therefore designers need a common model representation to share simulation models within a collaborative modeling environment.\cite{sinha_modeling_2001}\\
This approach has been used in the development of the Very High-Speed Integrated Circuit Hardware Description Language,
which has been used as a design automation tool in all phases of modern very large-scale integration design. 
Similarly, the U.S. Department of Defense and its contractors have used the High Level Architecture for simulation of
battlefield scenarios \cite{lutz_high_1998,park_relational_1994}.
Future modeling and simulation environments should allow for a tight integration
between application domains, either by interfacing the solvers or by using common model representations.
To further improve the exchange of model information, researchers have started to develop domain ontologies \cite{devedzic_survey_1999}. 
later, Ozawa et al. proposed a common ontology to support different levels of information sharing between humans and multiple
modeling and simulation software agents \cite{ozawa_model_2000}. 
Upon these domain ontologies, unified taxonomies and keyword networks
can be built to support model retrieval and repository management.\\
Methods, which aims to increase the amount of automation potential during the design process
are introduced in \cite{Herrmann2021, Berschik2021, Altun2021, Russwurm2021}.
This potential is shown in the figure \ref{pic:VDI2206}.
\begin{figure}[h]
    \centering
    \includegraphics[scale=0.4]{pics/VDI_2206.PNG}
    \caption{\label{pic:VDI2206} Automatisation potential in the VDI 2206 Guidline \cite{Jansch2006THEDO}.}
\end{figure}\\
If a complex multi-disciplinary systems can be modelled sufficiently, the question of the optimal solution rises.
A big issue that the system behavior, id hard to predict.
Therefore only optimizations approaches will be discussed, which can deal with systems,
where the system can be considered as a black box, excluding approaches like Gradient descent method.
Further it is important to make sure to find the global optimum and not to fall in local minima.
Therefor suitable approaches are: simulated annealing \cite{khachaturyan_thermodynamic_1981}, 
evolutionary algorithm \cite{wu_ensemble_2019}, particle swarm optimization \cite{Kennedy1995} and Bayesian optimization \cite{marcuk_optimization_1975}.\\
In \cite{hornby_automated_2006, khalafallah_electimize_2011, evans_aerodynamic_2017, slagter_perform_2020}
single domain based simulations have been combined with optimization methods.\\
However, the question remains, if a system can be designed to sufficiently describe the  
the causalities of a complex multi-disciplinary systems and if so, 
could it be used to optimize a complex product.
\section{Method}
From the beginning of the design process when each component, 
like part specific functions are briefly defined, the system interactions 
are described as a directed graph.
During further refinements in the design process, 
these components are changed by using methods like be 
subdivision, summariz or outsources.
The individual components therefore have to implement a certain interface, 
which allows to set the inputs and updates the outputs accordingly.

Using a directed graph method to represent each process Each step in a workflow can be represented as blackbox with a certain input and output (fig. \ref{pic:process-single}).
Due to simplicity the inputs and outputs have values that consists of a list of numbers. 
This values may be connected to other steps in the workflow resulting in different dependencies.
\begin{figure}[h]
    \centering
    \includegraphics[scale=0.5]{pics/VDI_2206.PNG}
    \caption{\label{pic:VDI2206} "General approach for development and construction (VDI 2221)." \cite{Jansch2006THEDO}}
\end{figure}\\
test
\begin{algorithm}[ht]
    \SetAlgoLined
    \KwIn{list of porcesses}
    \KwOut{sortet list of processes}
    use bubble sort to sort the list\; 
    \Fn{\FBSort{Process A, Process B}}{
        \eIf{A.InputConnections == B.InputConnections}{
            \eIf{A.Time == B.Time}{
                \eIf{A.InputConnections-A.InputChanges == B.InputConnections-B.InputChanges}{
                    \eIf{A.InputChanges == B.InputChanges}{
                        \Return{A.OutputConnections < B.OutputConnections}
                    }{
                        \Return{A.InputChanges > B.InputChanges}
                    }
                }{
                    \Return{A.InputConnections-A.InputChanges < B.InputConnections-B.InputChanges}
                }
            }{
                \Return{A.Time < B.Time}
            }
        }{
            \Return{A.InputConnections < B.InputConnections}
        }
    }
    \textbf{end}
\caption{\label{code:sort-process}Sort-Algorithm}
\end{algorithm}
text
\begin{algorithm}[bt]
    \KwIn{List of Processes\\
    Priority $\in$ ["manual", "dynamic", "time"]}
    \KwOut{Updated and converged processes}
    sort(List of Processes)\;
    \textbf{set} GState to "changed"\;
    \textbf{set} StartPos to 0\;
    \While{GState = "changed"}{
        \textbf{set} GState to "unchanged"\;
        \For{i = StartPos to sizeof(List of Processes)}{
            \textbf{set} CurrProcess as ith Element of List of Processes
            \textbf{set} GState to "unchanged"\;
            \textbf{set} STime to currentTime()\;
            \textbf{set} PState to CurrProcess.update()\;
            \textbf{set} ETime to currentTime()\;
            \textbf{set} $\Delta$Time to ETime - STime\;
            \If{Priority equals "time"}{
                CurrProcess.Time = round($\log_{10}${$\Delta$Time})\;
            }
            \If{PState equals "changed"}{
                \textbf{set} GState to "changed"\;
            }
            \If{CurrProcess.Inputs equals 0 \emph{\textbf{and}} Priority not "manuel"}{
                \textbf{set} StartPos to i\;
            }
            \If{Priority equals "dynamic" \emph{\textbf{or}} Priority equals "time"}{
                \textbf{set} List of FProcesses as List of Processes starting from CurrProcess\;
                \ForEach{FurtherProcess in  List of FProcesses}{
                    FurtherProcess.InputChanges.update()\;
                }
                sort(List of FProcesses)\;
            }
            increase i by one\;
       }
    }
\caption{\label{code:update-standard-processor}Update-Algorithm of the standard processor}
\end{algorithm}\\
Test
Test
test
\begin{figure}[h]
    \centering
    \includegraphics[scale=0.5]{pics/evol_alg.PNG}
    \caption{\label{pic:evol_alg} "General approach for development and construction (VDI 2221)." \cite{Jansch2006THEDO}}
\end{figure}\\
test

\section{Result}

\begin{figure}[h]
    \centering
    \includegraphics[scale=0.5]{pics/VDI_2206.PNG}
    \caption{\label{pic:VDI2206} "General approach for development and construction (VDI 2221)." \cite{Jansch2006THEDO}}
\end{figure}

ahnsdkjhauish
\input{src/discussion.tex}
\section{Conclusion}
Designing multifunctional and lightweight products is time consuming work, due to a high number of functional dependencies that have to be taken into account.
In this article a digital methodology is introduced, 
which aims to reduce the development time and 
increases the overall product quality by enabling multi domain optimizations.
The introduced system used the advantages of a complete digitizes, graph-based process representation.
It enables specialists to collaborate on problems with multiple domains and allows to outsource time-consuming operations to specialized servers.\\
With an heuristic approach the software tries to compute a stable configuration of the modeled interactions.
In the first example, this capability was used to solve the complex interactions of design problem which has to take the eigenweight into account.
It showed how different design process could be linked together, without introducing a new file format.\\
Additionally, given a ranking of multiple solution is described by a fitness function, the developed software can find the global optimum.
This was proven with the second example, where the system found the optimal layer constellation of an high-speed glass fiber reinforced plastics rotor.
It further demonstrated how multiple optimization goals can be taken into account.\\
While the test showed room for improvements in terms of performance, stability and user friendliness,
they also shows a major potential for great work relief and time savings.

\section*{List of Symbols}
\begin{table}[H]
    %\caption{\label{tb:los} A list of coordinates, which was used to generate the different degrees of deformation.}
    \centering
\begin{tabular}{c | c } 
    \toprule
    \textbf{symbol} & \textbf{meaning} \\ [0.5ex] 
    \midrule
    h & height\\
    J & Jacobian matrix\\
    m & Weibull module\\
    N & interpolation matrix\\
    NoE & number of elements\\
    NoN & number of nodes\\
    n & shape function\\
    $\sigma$ & stress\\
    V & Volume\\ 
    x, y & global coordinates\\
    $\xi$, $\eta$ & natural/local coordinates\\[1ex]
    \bottomrule
\end{tabular}
\end{table}

\section*{List of Indices}
\begin{table}[H]
    \centering
\begin{tabular}{c | c} 
    \toprule
\textbf{index} & \textbf{meaning} \\ [0.5ex] 
\midrule
$eff$ & effective\\
$f$ & rapture or failure\\
i & element number\\
j & node index in element\\
k & index of principle stress\\
$max$ & maximum\\ [1ex] 
\bottomrule
\end{tabular}
\end{table}

%\section*{acknowledgements}
%Acknowledgements should include contributions from anyone who does not meet the criteria for authorship (for example, to recognize contributions from people who provided technical help, collation of data, writing assistance, acquisition of funding, or a department chairperson who provided general support), as well as any funding or other support information.
%\printendnotes%???????????????????????????
% Submissions are not required to reflect the precise reference formatting of the journal (use of italics, bold etc.), however it is important that all key elements of each reference are included.
\bibliography{sources}
%\include{appendix}
\end{document}
