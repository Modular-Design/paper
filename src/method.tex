\section{Method}
In this paper the system used to describe the the interactions in a multi-domain
environment is a directed graph.
Therefore form the beginning of the design process when each component, 
like part specific functions are briefly defined, till to the cleaned up and summarized endproduct,
each step of the process have to described in the same manner.
This requires the ability to subdivide, summarize and outsource functionalities. 
Each function is blackbox with unknown system behavior, but known input and output values.
Therefor each system needs to implement a certain interface, 
which allows to control the system behavior with input data and options.
Additionaly, each step of the workflow needs an update mechanism to update the outputs accordingly.
The resulting entity with inputs, outputs, options and an update mechanism is further called a 'node'.\\
A system is build by connecting the output and input values of a two nodes.
The resulting dependencies can build complex interactions with multiple loops and multiple branches.
These systems are very similar to modern programming languages which are proven to be Turing-Complete.
This means, that no algorithm can be developed which can determine if the system will converege to a solution.
Therefore further approximations are introduced:
\begin{enumerate}
    \item the system does not converged, when after n solving runs the outputs values still change
    \item a output value is changed if the value differes from the previous update call
    \item a solver run is done if every node is updated at least once
\end{enumerate}

With this simplifications it is possible to uses the heuristic approach to approximate the solution.
However this approach requires a intelligent ranking of the nodes.
Therefor the following rules are introduced:
\begin{enumerate}
    \item nodes with no connected inputs are starting points and should always be at the beginning of a update progress
    \item nodes which needs more time to compute should update as rarely as possible
    \item nodes where a majority of all inputs have already changed have priority
    \item nodes with fewer connections have lesser dependencies and therefor have priority
\end{enumerate}
This rule based approach enables a fast deterministic updating process.\\
After solving all system dependencies multiple different permutations of the system inputs can be used to find a optimal system environment.
population based approaches as swarm and evolutionary based optimization methods are therefor recommended.
The fitness function is constructed from selected system properties.
To reduce the amount of recomputation the proposed system stores all generated systems as database.
The system is therfore capable of reusing the already computed variants and only computes missing systems.
