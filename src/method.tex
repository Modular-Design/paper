\section{Method}
The development of a methodology that supports the complex interactions of multifunctional designs can structured in:
\begin{itemize}
    \item model representation,
    \item system interfaces,
    \item solution strategy,
    \item additional capabilities.
\end{itemize}
As a sufficient and intuitive model representation the directed graph representation has been used for similar purposes.
Thats why the proposed system will also relay on this methodology 
to represent the interactions in a multi-domain environment.\\
The information that are exchanged between the two subsystems
are encoded in a scalable JSON format, which uses the HTTP-standard.
This approach allows to reuse many of the already used concepts to exchange complex information over the internet.
Additionally, it supports the client-server architecture, 
which supports that multiple experts can work together and
enables realtime information exchange in a multi-domain environment.\\
Further, the system is designed to be used in every step of a cyclic design approach.
This starts by inserting the conditions of the requirement list, 
over the functional abstraction and the synthesis to the automated validation of the developed solution.\\
This is manly archived by providing the ability to subdivide, summarize and outsource functionalities. 
The developed system is build on a black-box approach, making it able solve a wider variety of different problems.
Therefor each system only needs to implement a minimalistc interface, 
which allows to control the system behavior with certain, input driven characteristics and returning predefined output values.
The simplicity of the interface reduces the amount of work, when trying to implement custom solutions.
It further enables the user to easily share and implement their solutions like databases and simulation based evaluation nodes.\\
After representing the workflow, a solver has to find a suitable solution, by taking the interdependencies into account.
This requires, that each system implements its own update-mechanism to set the inputs of the subsystems.
A major issue, which arises from the complex interactions, is that by solving one system, changes to other systems can occure.
This means, that a solution is found, when a stable configuration, where no further changes are introduced, is reached.\\
Because finding a general solution is proven to be Turning-complete and therefore impossible to find deterministically, 
a set of rules to simplify this process are introduced:
\begin{itemize}
    \item the system has not converged, when after n solving runs the outputs values still change,
    \item an output value is changed if the value differs from the previous update call,
    \item a solver run is done if every node is updated at least once.
\end{itemize}
With this simplifications it is possible to uses the heuristic approach to approximate the solution.
However, this approach requires an intelligent ranking of the nodes.
For that purpose the following rules are introduced:
\begin{itemize}
    \item nodes with no connected inputs are starting points and should always be at the beginning of an update progress,
    \item nodes which need more time to compute should update as rarely as possible,
    \item nodes where a majority of all inputs have already changed have priority,
    \item nodes with fewer connections have less dependencies and therefor have priority.
\end{itemize}
This rule-based approach enables a fast deterministic updating process.\\
After solving all system dependencies, multiple different permutations of the system inputs can be used to find an optimal system environment.
Population based approaches as swarm and evolutionary based optimization methods are for that purpose recommended.
The fitness function is constructed from selected system properties.
To reduce the amount of recomputation time the proposed system stores the properties of all generated systems
in a database, which allows to reuse the already computed variants and only computes the missing systems.
This means, that for new designs that are based on the same design procedure and only differs by the used fitness function are far faster to find.