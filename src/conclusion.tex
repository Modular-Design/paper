\section{Conclusion}
Designing multifunctional and lightweight products is time consuming work, due to a high number of functional dependencies that have to be taken into account.
In this article a digital methodology is introduced, 
which aims to reduce the development time and 
increases the overall product quality by enabling multi domain optimizations.
The introduced system used the advantages of a complete digitizes, graph-based process representation.
It enables specialists to collaborate on problems with multiple domains and allows to outsource time-consuming operations to specialized servers.\\
With an heuristic approach the software tries to compute a stable configuration of the modeled interactions.
In the first example, this capability was used to solve the complex interactions of design problem which has to take the eigenweight into account.
It showed how different design process could be linked together, without introducing a new file format.\\
Additionally, given a ranking of multiple solution is described by a fitness function, the developed software can find the global optimum.
This was proven with the second example, where the system found the optimal layer constellation of an high-speed glass fiber reinforced plastics rotor.
It further demonstrated how multiple optimization goals can be taken into account.\\
While the test showed room for improvements in terms of performance, stability and user friendliness,
they also shows a major potential for great work relief and time savings.
