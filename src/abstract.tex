\begin{abstract}
A central process in composite lightweight engineering is the design of different fibre-reinforced parts. 
Some sub-processes such as the construction and numerical failure analysis 
as well as certain parts of the design process itself can be significantly improved 
using modern software tools. 
This often means that a compromise between optimization and 
increasing development costs must be found in order to balance 
structural complexity, number of design iteration loops and 
subsequent changes in the requirements.
We introduce a computer-driven automation process for a multi-domain, 
parameter-driven design optimization. 
The proposed concept was built around the idea that 
the methodology can be used with different software tools that 
are already in operation in the design process of lightweight structures and 
therefore allows an easy implementation in already existing development chains.
The developed process was successfully applied to different design scenarios, 
for example in designing of an high-speed glass fiber rotors with respect 
to their structural and dynamic performance. 
The results showed a significant decrease in time spending during the design phase with the benefit 
to quickly adapt the design to subsequent changes in the optimization goal.
Additionally, a wider solution space can be taken into account, which increases the quality of the optimization results.
%\keywords{Composites, Optimization, Design and Modelling, Smart by Design, Design Automation}
\end{abstract}